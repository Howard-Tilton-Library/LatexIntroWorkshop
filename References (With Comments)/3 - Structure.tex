% Each class has a number of options you can specify.
% Common options might include font size, number of columns, margins, type of paper, etc
\documentclass[11pt,letterpaper]{article}

% There are also many, many classes to choose from. (See slides for example output)
% Each one provides a different default styling, and occasionally new commands you can use
% to further customize your formatting.
%\documentclass[]{IEEEtran}
%\documentclass[]{proc}
%\documentclass[]{report}
%\documentclass[]{book}
%\documentclass[]{slides}
%\documentclass[]{memoir}
%\documentclass[]{letter}
%\documentclass[]{beamer}

% Here, we'll set up some variables that the article class uses.
\author{Your name here}

% If you don't specify a date, it will just use today's date.
\date{Enter name}

% You can use arbitrary LaTex markup inside of these commands.
% Here we're using a command to generate
% text that will be used for a title section, and inserting a manual line break.
\title{A \LaTeX document test}

% Later on, we'll call a command to make a title section (or even a title page)
% which will use these variables.
 
% The document is set up, so now we'll start writing actual content.
\begin{document}

% This command just gives us a nicely formatted title, using the variables we set earlier.
\maketitle

% This command adds a page break. 
\newpage

% To create a table of contents, add the command below
\tableofcontents

% Most document classes expect your content to be broken up into logical chunks. For the article
% class, it's sections, subsections, and so on.
% (For a book, you may use "chapter" commands and so on.)
\section{Introduction}
Here, you can just type normal text.

Create new paragraphs by pressing Enter twice.

% Notice that the compiler figures out the numbering for you
\section{Main Content}
The Latex engine also takes care of numbering your sections automatically. 

    % And the numbering is hierarchical.
    \subsection{Main Content Subsection}

        \subsubsection{Main Content Subsubsection}

        \subsubsection{Main Content Subsubsection 2}
        
    \subsection{Wrap Up}
    That's all for this section.


\section{Filler Text}
More document content!


\end{document}
